%  Template for scientific report
\documentclass{article}
\usepackage{spconf,amsmath,graphicx}
\usepackage[spanish]{babel}
\usepackage{placeins}
\usepackage{float}
\usepackage{makecell}
\usepackage{booktabs}
\usepackage{multirow}
\usepackage{hyperref}

% Title
\title{Estudio Empírico del Efecto del Tamaño de Población en un Algoritmo Evolutivo para el Problema de Ordenación Lineal}

\name{Lucas Molino Piñar}
\address{Escuela Técnica Superior de Ingeniería Informática, Universidad de Málaga\\
Bulevar Louis Pasteur, 35, Campus de Teatinos, 29017 Málaga}

\begin{document}
\maketitle

\begin{abstract}
Este trabajo presenta un estudio empírico sobre la influencia del tamaño de población en el rendimiento de un Algoritmo Evolutivo (AE) aplicado al Problema de Ordenación Lineal (LOP). Se implementó un AE con representación de permutaciones, cruce PMX y mutación por intercambio, evaluando cinco tamaños de población (10, 20, 50, 100, 200) sobre diez instancias del benchmark LOLIB. Cada configuración se ejecutó 30 veces con 10,000 evaluaciones por ejecución, totalizando 1,500 experimentos. Los resultados fueron analizados mediante tests estadísticos no paramétricos (Kruskal-Wallis y Mann-Whitney U con corrección de Bonferroni). Los experimentos muestran que para cuatro de las diez instancias existen diferencias estadísticamente significativas entre tamaños de población, aunque el efecto práctico es limitado. El gap medio respecto al óptimo conocido varía entre 2-4\% para todos los tamaños de población evaluados.
\end{abstract}

\begin{keywords}
Algoritmo Evolutivo, Problema de Ordenación Lineal, Optimización Combinatoria, Análisis Empírico
\end{keywords}

\section{Introducción}
\label{sec:intro}

El Problema de Ordenación Lineal es un problema de optimización combinatoria NP-difícil con aplicaciones en diseño industrial, análisis input-output económico y arqueología \cite{schiavinotto2004}. Formalmente, dada una matriz cuadrada $\mathbf{A} = (a_{ij})$ de dimensión $n \times n$, el objetivo es encontrar una permutación $\sigma$ de las filas y columnas que maximice la suma de los elementos situados por encima de la diagonal principal:

\begin{equation}
f(\sigma) = \sum_{i=1}^{n-1} \sum_{j=i+1}^{n} a_{\sigma(i), \sigma(j)}
\label{eq:lop}
\end{equation}

Los Algoritmos Evolutivos (AEs) han demostrado ser eficaces para resolver el LOP \cite{laguna1999}, aunque su rendimiento depende críticamente de varios parámetros, siendo el tamaño de población uno de los más influyentes. Un tamaño reducido puede llevar a convergencia prematura, mientras que uno excesivo incrementa el coste computacional sin garantizar mejoras proporcionales.

El objetivo de este estudio es analizar empíricamente la influencia del tamaño de población en el rendimiento de un AE aplicado al LOP, evaluando cinco configuraciones sobre instancias del benchmark LOLIB y determinando mediante análisis estadístico riguroso si existen diferencias significativas en la calidad de las soluciones obtenidas.

\section{Metodología}
\label{sec:methodology}

\subsection{Algoritmo Evolutivo Implementado}

Se implementó un AE generacional con elitismo y los siguientes componentes:

\textbf{Representación:} Permutación directa de enteros $[0, n-1]$, donde la posición $i$ indica el índice de la fila/columna en la posición $i$ de la ordenación.

\textbf{Inicialización:} Generación aleatoria uniforme de permutaciones válidas.

\textbf{Función de evaluación:} Ecuación \ref{eq:lop}, implementada con complejidad $O(n^2)$ mediante doble bucle sobre el triángulo superior.

\textbf{Selección:} Torneo de tamaño 3 para selección de padres.

\textbf{Cruce:} Partially Mapped Crossover (PMX) con probabilidad $p_c = 0.9$, seleccionando aleatoriamente dos puntos de corte e intercambiando el segmento intermedio con mapeo de conflictos.

\textbf{Mutación:} Intercambio aleatorio de dos posiciones (swap mutation) con probabilidad $p_m = 0.1$.

\textbf{Reemplazo:} Generacional con elitismo, preservando el mejor individuo de la generación anterior.

\textbf{Criterio de parada:} 10,000 evaluaciones de la función objetivo.\footnote{El código completo de la implementación está disponible en: \url{https://github.com/USUARIO/REPOSITORIO}}

\subsection{Diseño Experimental}

Para este estudio, se evaluaron cinco tamaños de población: $N \in \{10, 20, 50, 100, 200\}$ sobre diez instancias del benchmark LOLIB: be75eec, stabu1, t59b11xx, t65b11xx, t69r11xx, t70b11xx, t74d11xx, t75d11xx, tiw56n54, tiw56n72. Para cada combinación instancia-tamaño se realizaron 30 ejecuciones independientes con semillas aleatorias diferentes pero determinísticas (base seed + run\_id), totalizando 1,500 experimentos.

Se registraron el mejor valor de fitness, la solución correspondiente y datos de convergencia en cada ejecución. Los resultados fueron analizados mediante:

\begin{itemize}
\item Estadísticos descriptivos (media, desviación típica, mediana, mínimo, máximo)
\item Gap porcentual respecto al óptimo conocido: $gap = \frac{f_{opt} - \bar{f}}{f_{opt}} \times 100$
\item Test de Kruskal-Wallis ($\alpha = 0.05$) para detectar diferencias globales entre tamaños
\item Tests pareados de Mann-Whitney U con corrección de Bonferroni ($\alpha' = 0.05/10 = 0.005$) para comparaciones específicas
\end{itemize}

\section{Resultados}
\label{sec:results}

\subsection{Análisis Descriptivo}

La Tabla~\ref{tab:summary} presenta los resultados agregados para una muestra representativa de instancias. En general, todos los tamaños de población obtienen soluciones de calidad similar, con gaps del 2-4\% respecto a los óptimos conocidos. Los resultados estadísticos completos para las 50 configuraciones se presentan en el Anexo~B.

\begin{table}[h]
\centering
\caption{Resultados estadísticos para instancias seleccionadas}
\label{tab:summary}
\small
\begin{tabular}{@{}llrrr@{}}
\toprule
\textbf{Instancia} & \textbf{Pop} & \textbf{Media} & \textbf{Desv.} & \textbf{Gap (\%)} \\
\midrule
\multirow{5}{*}{be75eec} 
& 10  & 257194 & 3064 & 2.92 \\
& 20  & 258810 & 1905 & 2.31 \\
& 50  & 257894 & 2602 & 2.66 \\
& 100 & 258749 & 1840 & 2.34 \\
& 200 & 258026 & 1822 & 2.61 \\
\midrule
\multirow{5}{*}{stabu1}
& 10  & 408919 & 3296 & 3.12 \\
& 20  & 409547 & 2525 & 2.97 \\
& 50  & 408308 & 3546 & 3.26 \\
& 100 & 409510 & 2361 & 2.98 \\
& 200 & 406611 & 2831 & 3.67 \\
\midrule
\multirow{5}{*}{tiw56n54}
& 10  & 108627 & 1448 & 3.67 \\
& 20  & 109248 & 1036 & 3.12 \\
& 50  & 109327 &  775 & 3.05 \\
& 100 & 109010 &  900 & 3.33 \\
& 200 & 108632 & 1121 & 3.67 \\
\bottomrule
\end{tabular}
\end{table}

La Figura~\ref{fig:convergence} muestra las curvas de convergencia para la instancia be75eec, representando tanto el mejor fitness alcanzado (panel izquierdo) como el fitness promedio de la población (panel derecho) en función del número de evaluaciones. 

\begin{figure*}[t]
\centering
\includegraphics[width=0.95\textwidth]{results/figures/convergence_be75eec.png}
\caption{Curvas de convergencia para la instancia be75eec. Panel izquierdo: evolución del mejor individuo encontrado. Panel derecho: evolución del fitness promedio poblacional. Cada curva representa un tamaño de población diferente.}
\label{fig:convergence}
\end{figure*}

El análisis de estas curvas revela varios aspectos relevantes del comportamiento del algoritmo. En primer lugar, se observa que las poblaciones de mayor tamaño (100 y 200 individuos) exhiben una convergencia inicial más rápida, alcanzando valores de fitness superiores a 240,000 en las primeras 1,000 evaluaciones. Por el contrario, la población de 10 individuos presenta la convergencia más lenta en la fase inicial, manifestando un comportamiento más errático con saltos discontinuos, característico de poblaciones pequeñas con alta variabilidad estocástica.

Sin embargo, al analizar la convergencia a largo plazo (más allá de 5,000 evaluaciones), todas las configuraciones tienden a estabilizarse en valores similares, convergiendo hacia un rango de fitness entre 257,000 y 261,000. Este patrón sugiere que, aunque el tamaño de población influye en la velocidad de convergencia inicial, su impacto en la calidad final de la solución es limitado dado el presupuesto computacional fijo.

El panel derecho, que muestra el fitness promedio poblacional, permite observar la diversidad del algoritmo. Las poblaciones grandes mantienen promedios más bajos durante más tiempo, indicando mayor diversidad, mientras que las poblaciones pequeñas convergen rápidamente hacia el mejor individuo, evidenciando menor capacidad exploratoria. Este comportamiento es consistente con la teoría de AEs: poblaciones pequeñas favorecen la explotación, mientras que poblaciones grandes mantienen la exploración por más generaciones.

\subsection{Análisis Estadístico}

La Tabla~\ref{tab:kruskal} resume los resultados del test de Kruskal-Wallis. Solo cuatro de las diez instancias muestran diferencias estadísticamente significativas ($p < 0.05$) entre tamaños de población donde \textit{ns} es no significativo; *: $p<0.05$; **: $p<0.01$; ***: $p<0.001$

\begin{table}[h]
\centering
\caption{Resultados del test de Kruskal-Wallis}
\label{tab:kruskal}
\begin{tabular}{@{}lrrl@{}}
\toprule
\textbf{Instancia} & \textbf{H} & \textbf{p-valor} & \textbf{Sig.} \\
\midrule
be75eec   & 7.92  & 0.094 & ns \\
stabu1    & 19.71 & 0.001 & *** \\
t59b11xx  & 1.35  & 0.853 & ns \\
t65b11xx  & 11.95 & 0.018 & ** \\
t69r11xx  & 3.28  & 0.512 & ns \\
t70b11xx  & 4.86  & 0.302 & ns \\
t74d11xx  & 5.11  & 0.276 & ns \\
t75d11xx  & 11.97 & 0.018 & ** \\
tiw56n54  & 9.87  & 0.043 & * \\
tiw56n72  & 2.15  & 0.708 & ns \\
\bottomrule
\multicolumn{4}{c}{}
\end{tabular}
\end{table}

Los resultados del test de Kruskal-Wallis muestran un panorama heterogéneo respecto a la influencia del tamaño de población. De las diez instancias evaluadas, seis (be75eec, t59b11xx, t69r11xx, t70b11xx, t74d11xx, tiw56n72) no presentan diferencias estadísticamente significativas ($p > 0.05$), con p-valores que van desde 0.094 hasta 0.853. Este resultado indica que para estas instancias, la variabilidad intra-grupo (dentro de cada tamaño de población) es comparable o superior a la variabilidad inter-grupo (entre diferentes tamaños), sugiriendo que el tamaño de población no constituye un factor determinante del rendimiento.

Particularmente notable es el caso de t59b11xx ($H = 1.35, p = 0.853$), que exhibe el p-valor más alto del estudio. El análisis de los datos revela que esta instancia mantiene medias extremadamente similares para todos los tamaños (239,303 - 239,918), con desviaciones típicas consistentemente altas (2,259 - 3,652). Este patrón sugiere que la estructura específica de esta instancia genera un paisaje de búsqueda donde la estocasticidad del algoritmo domina sobre el efecto del parámetro poblacional.

En contraste, cuatro instancias (stabu1, t65b11xx, t75d11xx, tiw56n54) sí muestran diferencias significativas. El caso más extremo es \textbf{stabu1} ($H = 19.71, p = 0.001$), con un p-valor tres órdenes de magnitud por debajo del umbral de significancia. Examinando los datos, se observa que esta instancia presenta la mayor amplitud en medias entre tamaños de población: desde 406,611 (pop. 200) hasta 409,547 (pop. 20), representando una diferencia de 2,936 puntos. Además, la población de 200 individuos exhibe sistemáticamente peor rendimiento que configuraciones más pequeñas, fenómeno que se discutirá en detalle en las comparaciones pareadas.

Las instancias \textbf{t65b11xx} y \textbf{t75d11xx} muestran significancia moderada ($p = 0.018$ ambas) con estadísticos H similares (11.95 y 11.97 respectivamente). Para t65b11xx, la población de 100 individuos alcanza una media de 401,049, claramente superior a las poblaciones extremas (10: 398,375; 200: 397,751). Este patrón en forma de U invertida sugiere la existencia de un óptimo poblacional en el rango intermedio-alto. En t75d11xx se observa un comportamiento similar, donde poblaciones intermedias (20, 50, 100) superan a las extremas (10, 200).

Finalmente, \textbf{tiw56n54} presenta significancia marginal ($H = 9.87, p = 0.043$), apenas por debajo del umbral $\alpha = 0.05$. Los datos muestran un patrón curioso: las poblaciones extremas (10 y 200) coinciden exactamente en media (108,627 y 108,632), mientras que tamaños intermedios alcanzan valores ligeramente superiores (109,248 - 109,327). La proximidad del p-valor al límite de significancia y la pequeña magnitud de las diferencias (menos del 1\% de variación) sugieren que este resultado debe interpretarse con cautela.

Para las instancias con diferencias significativas, los tests pareados de Mann-Whitney U (Tabla~\ref{tab:pairwise}) permiten identificar qué pares de poblaciones difieren específicamente. En stabu1 y t65b11xx, las poblaciones más pequeñas o medianas (10, 20, 100) superan significativamente a la población de 200 individuos.

\begin{table}[h]
\centering
\caption{Comparaciones pareadas significativas (Mann-Whitney U)}
\label{tab:pairwise}
\small
\begin{tabular}{@{}llrl@{}}
\toprule
\textbf{Instancia} & \textbf{Comparación} & \textbf{p-valor} & \textbf{Mejor} \\
\midrule
\multirow{3}{*}{stabu1}
& 10 vs 200  & 0.0044 & 10 \\
& 20 vs 200  & 0.0001 & 20 \\
& 100 vs 200 & 0.0000 & 100 \\
\midrule
t65b11xx & 100 vs 200 & 0.0003 & 100 \\
\bottomrule
\end{tabular}
\end{table}

\subsection{Análisis del Gap}

La Figura~\ref{fig:heatmap} presenta un mapa de calor del gap porcentual respecto al óptimo conocido, permitiendo visualizar simultáneamente el rendimiento de todas las configuraciones evaluadas. 

\begin{figure*}[t]
\centering
\includegraphics[width=0.85\textwidth]{results/figures/gap_heatmap.png}
\caption{Mapa de calor del gap porcentual respecto al óptimo conocido. La escala de color codifica la distancia al óptimo: verde oscuro indica mejor rendimiento (menor gap), mientras que rojo oscuro indica peor rendimiento (mayor gap). Los valores numéricos muestran el gap exacto en porcentaje.}
\label{fig:heatmap}
\end{figure*}

El análisis detallado del heatmap revela varios patrones significativos. En primer lugar, se observa una marcada heterogeneidad en la dificultad de las instancias: las instancias be75eec, t59b11xx, t69r11xx, t70b11xx, t74d11xx y t75d11xx presentan gaps en el rango 2.1-3.2\%, caracterizándose como instancias de dificultad baja a moderada para el AE implementado. Estas instancias muestran consistentemente coloración verde, indicando soluciones de alta calidad independientemente del tamaño de población.

En contraste, las instancias stabu1, t65b11xx, tiw56n54 y tiw56n72 exhiben gaps superiores, alcanzando hasta 3.7\% en los casos menos favorables. Particularmente notable es el caso de tiw56n72, que presenta la fila más rojiza del mapa con gaps entre 3.18-3.58\%, sugiriendo características estructurales que dificultan la optimización mediante el AE propuesto. La instancia tiw56n54 también destaca por su uniformidad en el rango 3.05-3.67\%, sin mostrar un tamaño de población claramente superior.

Respecto al efecto del tamaño de población, el heatmap confirma cuantitativamente la limitada variabilidad horizontal observada en los tests estadísticos. Para la mayoría de instancias, la diferencia entre el mejor y peor tamaño de población no supera 0.6 puntos porcentuales. Sin embargo, se identifican tres excepciones relevantes:

\textbf{Stabu1} (fila 2): Presenta la mayor variabilidad, con gaps desde 2.97\% (población 20, verde) hasta 3.67\% (población 200, rojo oscuro). Este patrón sugiere que poblaciones intermedias son más efectivas que poblaciones grandes para esta instancia particular, posiblemente debido a un balance óptimo entre exploración y explotación dado el presupuesto computacional.

\textbf{T65b11xx} (fila 4): Muestra un gradiente ascendente parcial desde población 100 (2.59\%, verde) hacia poblaciones extremas (3.24-3.40\%, naranja), indicando que poblaciones intermedias-grandes ofrecen el mejor compromiso para esta instancia.

\textbf{T70b11xx} (fila 6): Destaca por presentar el mejor rendimiento individual del estudio con población 20 (2.11\%, verde oscuro), contrastando con el rendimiento inferior de poblaciones pequeñas (10) y grandes (200), evidenciando un óptimo claro en tamaños intermedios.

Un hallazgo relevante es la ausencia de un patrón columnar consistente: ninguna columna (tamaño de población) es sistemáticamente verde o roja, confirmando que el tamaño óptimo es fuertemente dependiente de la instancia. Este resultado contradice la hipótesis inicial de que poblaciones grandes serían uniformemente superiores, sugiriendo en cambio que la interacción entre estructura del problema y parámetros del algoritmo es compleja y no lineal.

Finalmente, el rango global de gaps (2.11-3.67\%) indica que todas las configuraciones alcanzan soluciones de calidad razonable, con diferencias prácticas limitadas. Esta compresión del rango de valores es típica en metaheurísticas bien ajustadas y sugiere que otros parámetros (tasas de operadores, presupuesto computacional) podrían tener mayor impacto que el tamaño de población en el rendimiento global del algoritmo. Para un análisis complementario de la distribución de resultados mediante diagramas de caja, consúltese el Anexo~A.

\section{Conclusiones}
\label{sec:conclusions}

Este estudio empírico ha evaluado sistemáticamente el efecto del tamaño de población en un AE para el LOP mediante 1,500 experimentos controlados. Los resultados principales son:

\begin{enumerate}
\item El tamaño de población no exhibe un efecto consistente en todas las instancias del LOP. Solo en 4 de 10 instancias se detectaron diferencias estadísticamente significativas.

\item Cuando existen diferencias, los resultados son contradictorios: en algunos casos poblaciones pequeñas (10, 20) superan a las grandes (200), sugiriendo que poblaciones grandes pueden sufrir de exceso de exploración sin explotación adecuada dado el presupuesto computacional fijo.

\item El efecto práctico del tamaño de población es limitado: el gap respecto al óptimo varía menos de 1 punto porcentual entre configuraciones para una misma instancia.

\item Todos los tamaños evaluados alcanzan gaps del 2-4\%, indicando que el AE implementado es razonablemente efectivo independientemente del tamaño de población dentro del rango estudiado.
\end{enumerate}

Como trabajo futuro, sería interesante evaluar la interacción del tamaño de población con otros parámetros (tasas de mutación/cruce), considerar presupuestos computacionales variables en lugar de fijos, y analizar la diversidad poblacional durante la ejecución para comprender mejor los mecanismos subyacentes.

\section*{Declaración sobre IA Generativa}
Durante la preparación de este trabajo, el autor empleó IA generativa para revisiones gramaticales, ortográficas y apoyo en la implementación del código experimental. El autor revisó y editó todo el contenido, asumiendo plena responsabilidad del trabajo presentado.


\bibliographystyle{IEEEbib}
\begin{thebibliography}{1}

\bibitem{schiavinotto2004}
T. Schiavinotto and T. Stützle,
\newblock ``The linear ordering problem: Instances, search space analysis and algorithms,''
\newblock \emph{Journal of Mathematical Modelling and Algorithms}, vol. 3, pp. 367--402, 2004.

\bibitem{laguna1999}
M. Laguna, R. Martí, and V. Campos,
\newblock ``Intensification and diversification with elite tabu search solutions for the linear ordering problem,''
\newblock \emph{Computers \& Operations Research}, vol. 26, no. 12, pp. 1217--1230, 1999.

\end{thebibliography}

\newpage
\onecolumn

\appendix

\section{Diagramas de Caja por Instancia}
\label{appendix:boxplots}

La Figura~\ref{fig:boxplots} presenta los diagramas de caja (boxplots) de la distribución de resultados para las diez instancias evaluadas, permitiendo visualizar simultáneamente la tendencia central, dispersión y presencia de valores atípicos.

\begin{figure}[h]
\centering
\includegraphics[width=0.95\textwidth]{results/figures/boxplots_by_instance.png}
\caption{Diagramas de caja de la distribución de fitness para cada combinación de instancia y tamaño de población (30 repeticiones por configuración). La línea roja discontinua indica el óptimo conocido. Los círculos representan valores atípicos (outliers).}
\label{fig:boxplots}
\end{figure}

El análisis de los boxplots confirma varios hallazgos previamente discutidos. En primer lugar, se observa que la mayoría de las instancias presentan distribuciones con medianas (línea central de la caja) relativamente estables entre tamaños de población, con cajas de altura similar, indicando dispersiones comparables. Este patrón es especialmente evidente en t59b11xx y t69r11xx, donde las cajas son prácticamente indistinguibles entre configuraciones, corroborando la ausencia de diferencias significativas detectada por el test de Kruskal-Wallis.

Las instancias con diferencias significativas muestran patrones distintivos. En \textbf{stabu1}, la población de 200 individuos presenta una distribución claramente desplazada hacia abajo respecto a las demás, con una mediana inferior y mayor presencia de outliers en la zona baja. En \textbf{t65b11xx}, la población de 100 muestra la caja más elevada y compacta, mientras que las poblaciones extremas (10, 200) exhiben mayor dispersión y presencia de valores atípicos.

Un aspecto relevante es la variabilidad del número de outliers entre instancias. Las instancias más pequeñas (tiw56n54, tiw56n72) presentan pocos o ningún valor atípico, mientras que instancias de mayor dimensión (t70b11xx, t74d11xx, t75d11xx) muestran mayor frecuencia de outliers, sugiriendo que el aumento de la dimensionalidad del problema incrementa la variabilidad estocástica del algoritmo. La distancia de todas las distribuciones respecto a la línea roja (óptimo conocido) visualiza directamente el gap analizado en la Sección~\ref{sec:results}, confirmando que ninguna configuración alcanza sistemáticamente el óptimo global.


\newpage

\section{Resultados Estadísticos Completos}
\label{appendix:complete_results}

La Tabla~\ref{tab:complete_results} presenta los resultados estadísticos completos para las 50 configuraciones experimentales (10 instancias × 5 tamaños de población), incluyendo media, desviación típica, mejor y peor valor obtenido, mediana, óptimo conocido y gap porcentual.

\begin{table}[h]
\centering
\caption{Resultados estadísticos completos (30 runs por configuración)}
\label{tab:complete_results}
\scriptsize
\begin{tabular}{@{}lrrrrrrr@{}}
\toprule
\textbf{Inst.} & \textbf{Pop} & \textbf{Media} & \textbf{Desv.} & \textbf{Mejor} & \textbf{Peor} & \textbf{Ópt.} & \textbf{Gap (\%)} \\
\midrule
be75eec   & 10  & 257194 & 3013 & 261639 & 248842 & 264940 & 2.92 \\
be75eec   & 20  & 258810 & 1873 & 261769 & 254269 & 264940 & 2.31 \\
be75eec   & 50  & 257894 & 2558 & 261182 & 250273 & 264940 & 2.66 \\
be75eec   & 100 & 258749 & 1809 & 261609 & 254832 & 264940 & 2.34 \\
be75eec   & 200 & 258026 & 1792 & 261104 & 253030 & 264940 & 2.61 \\
\midrule
stabu1    & 10  & 408919 & 3241 & 414655 & 401613 & 422088 & 3.12 \\
stabu1    & 20  & 409547 & 2483 & 414575 & 403233 & 422088 & 2.97 \\
stabu1    & 50  & 408308 & 3487 & 413681 & 399015 & 422088 & 3.26 \\
stabu1    & 100 & 409510 & 2321 & 413352 & 402128 & 422088 & 2.98 \\
stabu1    & 200 & 406611 & 2783 & 410915 & 399985 & 422088 & 3.67 \\
\midrule
t59b11xx  & 10  & 239919 & 3213 & 243615 & 230975 & 245750 & 2.37 \\
t59b11xx  & 20  & 239652 & 3590 & 243245 & 228685 & 245750 & 2.48 \\
t59b11xx  & 50  & 239303 & 3179 & 243945 & 232650 & 245750 & 2.62 \\
t59b11xx  & 100 & 239432 & 3325 & 244325 & 230470 & 245750 & 2.57 \\
t59b11xx  & 200 & 239756 & 2259 & 243505 & 233705 & 245750 & 2.44 \\
\midrule
t65b11xx  & 10  & 398375 & 4390 & 405828 & 388234 & 411733 & 3.24 \\
t65b11xx  & 20  & 399240 & 4420 & 405009 & 387402 & 411733 & 3.03 \\
t65b11xx  & 50  & 399413 & 4156 & 406161 & 391416 & 411733 & 2.99 \\
t65b11xx  & 100 & 401049 & 3105 & 405088 & 393266 & 411733 & 2.59 \\
t65b11xx  & 200 & 397751 & 3527 & 403218 & 389382 & 411733 & 3.40 \\
\midrule
t69r11xx  & 10  & 843649 & 6681 & 855536 & 827714 & 865650 & 2.54 \\
t69r11xx  & 20  & 844778 & 6185 & 855525 & 827272 & 865650 & 2.41 \\
t69r11xx  & 50  & 844677 & 7544 & 857712 & 829720 & 865650 & 2.42 \\
t69r11xx  & 100 & 844374 & 6408 & 860806 & 831115 & 865650 & 2.46 \\
t69r11xx  & 200 & 842264 & 5895 & 852976 & 830785 & 865650 & 2.70 \\
\midrule
t70b11xx  & 10  & 606488 & 9020 & 619390 & 584516 & 623411 & 2.71 \\
t70b11xx  & 20  & 610280 & 6760 & 619735 & 590232 & 623411 & 2.11 \\
t70b11xx  & 50  & 607435 & 7147 & 617057 & 593472 & 623411 & 2.56 \\
t70b11xx  & 100 & 609200 & 4332 & 616610 & 599994 & 623411 & 2.28 \\
t70b11xx  & 200 & 607767 & 5115 & 614591 & 592624 & 623411 & 2.51 \\
\midrule
t74d11xx  & 10  & 655599 & 6629 & 665376 & 633055 & 673346 & 2.64 \\
t74d11xx  & 20  & 655257 & 6775 & 666129 & 641081 & 673346 & 2.69 \\
t74d11xx  & 50  & 653990 & 6921 & 661476 & 633981 & 673346 & 2.87 \\
t74d11xx  & 100 & 653706 & 7418 & 664590 & 633197 & 673346 & 2.92 \\
t74d11xx  & 200 & 652837 & 5079 & 659699 & 641463 & 673346 & 3.05 \\
\midrule
t75d11xx  & 10  & 666674 & 8559 & 679827 & 639964 & 688601 & 3.18 \\
t75d11xx  & 20  & 671135 & 4786 & 679391 & 661668 & 688601 & 2.54 \\
t75d11xx  & 50  & 669867 & 6326 & 683531 & 651719 & 688601 & 2.72 \\
t75d11xx  & 100 & 671181 & 6556 & 683402 & 658221 & 688601 & 2.53 \\
t75d11xx  & 200 & 667290 & 4588 & 675975 & 657486 & 688601 & 3.09 \\
\midrule
tiw56n54  & 10  & 108627 & 1424 & 110898 & 103623 & 112767 & 3.67 \\
tiw56n54  & 20  & 109248 & 1019 & 110826 & 107086 & 112767 & 3.12 \\
tiw56n54  & 50  & 109327 &  761 & 110323 & 107135 & 112767 & 3.05 \\
tiw56n54  & 100 & 109010 &  885 & 110360 & 106537 & 112767 & 3.33 \\
tiw56n54  & 200 & 108632 & 1102 & 110510 & 106309 & 112767 & 3.67 \\
\midrule
tiw56n72  & 10  & 446402 & 4603 & 453648 & 431021 & 462991 & 3.58 \\
tiw56n72  & 20  & 447172 & 4133 & 456489 & 435236 & 462991 & 3.42 \\
tiw56n72  & 50  & 447579 & 4687 & 457143 & 435937 & 462991 & 3.33 \\
tiw56n72  & 100 & 448284 & 3324 & 455647 & 442497 & 462991 & 3.18 \\
tiw56n72  & 200 & 447264 & 3473 & 452986 & 439079 & 462991 & 3.40 \\
\bottomrule
\end{tabular}
\end{table}

\end{document}
